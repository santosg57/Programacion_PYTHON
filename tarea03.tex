
\documentclass[11pt]{article}

% Esto es para poder escribir acentos directamente:
\usepackage[latin1]{inputenc}
% Esto es para que el LaTeX sepa que el texto est� en espa�ol:
\usepackage[spanish]{babel}
\spanishdecimal{.}
\usepackage{color}

\usepackage{geometry}
 \geometry{
 a4paper,
 total={170mm,257mm},
 left=20mm,
 top=20mm,
 }
 
% Paquetes de la AMS:
\usepackage{amsmath, amsthm, amsfonts}

% Teoremas
%--------------------------------------------------------------------------
\newtheorem{thm}{Teorema}[section]
\newtheorem{cor}[thm]{Corolario}
\newtheorem{lem}[thm]{Lema}
\newtheorem{prop}[thm]{Proposici�n}
\theoremstyle{definition}
\newtheorem{defn}[thm]{Definici�n}
\theoremstyle{remark}
\newtheorem{rem}[thm]{Observaci�n}

% Atajos.
% Se pueden definir comandos nuevos para acortar cosas que se usan
% frecuentemente. Como ejemplo, aqu� se definen la R y la Z dobles que
% suelen representar a los conjuntos de n�meros reales y enteros.
%--------------------------------------------------------------------------

\def\RR{\mathbb{R}}
\def\ZZ{\mathbb{Z}}

% De la misma forma se pueden definir comandos con argumentos. Por
% ejemplo, aqu� definimos un comando para escribir el valor absoluto
% de algo m�s f�cilmente.
%--------------------------------------------------------------------------
\newcommand{\abs}[1]{\left\vert#1\right\vert}

% Operadores.
% Los operadores nuevos deben definirse como tales para que aparezcan
% correctamente. Como ejemplo definimos en jacobiano:
%--------------------------------------------------------------------------
\DeclareMathOperator{\Jac}{Jac}

%--------------------------------------------------------------------------
\title{Introducci\'on a Lenguaje de Programaci\'on \\
PYTHON}
\author{Juan Per\'ez\\
  \small Dept. de Computo\\
  \small Instituto de Computo\\
  \small juanito@gmail.com
}

\begin{document}
%\maketitle

%\abstract{Esto es una plantilla simple para un art�culo en \LaTeX.}

\textbf{TAREA02}

\vspace{.5mm}

CURSO: Introducci\'on al Lenguaje de Programaci\'on PYTHON

\vspace{.5 mm}

Profesor: Leopoldo Gonz\'alez-Santos \hspace{5cm} Correo: \textbf{lgs@unam.mx}

\vspace{.5 mm}

Fecha: \today

\hrulefill

\begin{enumerate}
\item Encuentre la recta que mejor se ajuste a los puntos dados  por el m\'etodo de m\'inimos cuadrados. Los puntos son \{(1,1.3), (2, 1.8), (3, 3.4), (4, 3.8), (5., 5.4), (6, 6.3), (7, 6.5), (8., 8), (9, 9), (10,10.4)\}. Graficar los puntos y la recta, entregar los resultados en formato pdf.

La regresi\'on lineal simple esta dada como: $y = \beta_o + \beta_1 x$ donde 

\[
\beta_1 = \frac{\sum_1^n (x_i - \bar{x}) (y_i - \bar{y})} {\sum_1^n (x_i - \bar{x})^2}
\]

y

\[
\beta_o = \bar{y} - \beta_1 \bar{x}
\]

\item Encuentre la correlaci\'on lineal del conjunto de puntos del problema anterior.  \cite{corr}

La correlaci\'on lineal esta dada por

\[
r = \frac{\sum_1^n (x_i - \bar{x}) (y_i - \bar{y})} {\sqrt{\sum_1^n (x_i - \bar{x})^2}  \sqrt{\sum_1^n (y_i - \bar{y})^2}}
\]

\item Grafique de manera aleatoria 100 c\'irculos dentro del \'area [-10, 10] x [-10, 10], con radio aleatorio entre .2 y 1. 

\item  Grafique la campana de Gauss, dada por: \cite{campana}

\[
f(x) = \frac{1}{\sigma \sqrt{2 \pi}} e^{- \frac{(x -\mu)^2}{2 \sigma^2}},  \hspace{.5cm} \text{en} \hspace{3mm} ( -\infty, \infty)
\]

tomado $\mu = 50$ y $\sigma = 10$.


\end{enumerate}

\hrulefill

\textbf{\color{red} NOTA.} Enviar los resultados y programas en un archivo en formato {\color{blue} pdf}.

% Bibliograf�a.
%-----------------------------------------------------------------
\begin{thebibliography}{99}

\bibitem{Cd94} Autor, \emph{T�tulo}, Revista/Editor, (a�o)
\bibitem{corr} wikipedia, https://es.wikipedia.org/wiki/Correlaci\'on
\bibitem{campana} https://www.ecured.cu/Campana\_de\_Gauss


\end{thebibliography}

\end{document}