\documentclass{beamer}

% https://es.wikipedia.org/wiki/Sistema_de_numeraci%C3%B3n

\usetheme{Boadilla}
\usepackage{lmodern}
\usepackage{amsmath}
\usepackage{tikz}
\usepackage{amssymb}
\usetikzlibrary{trees}

% Instituto de Neurobiolog\'ia \\ UNAM - M\'exico

\newcommand\nombre{Juan Per\'ez}
\newcommand\depto{Computaci\'on}
\newcommand\instituto{Instituto de Computaci\'on}
\newcommand\correo{juanito@gmail.com}
\newcommand\iniciales{COMPU}

\newtheorem{mytheorem}{Theorem}
\newtheorem{mylemma}{Lemma}
\newtheorem{mycorollary}{Corollary}
\theoremstyle{definition}
\newtheorem{mydefinition}{Definition}
\newtheorem{historylabel}{Historycal label}
\newtheorem{myexample}{Example}
\theoremstyle{remark}
\newtheorem{remark}{Remark}

\newcommand{\bR}{{\mathbf{R}}}
\newcommand{\bC}{{\mathbf{C}}}
\newcommand{\bT}{{\mathbf{T}}}
\newcommand{\bZ}{{\mathbf{Z}}}
\newcommand{\myemph}[1]{\alert{#1}}
\newcommand{\essinf}{\mathop{\mathrm{ess\,inf}}}
\renewcommand{\phi}{\varphi}
\newcommand{\eps}{\varepsilon}
\DeclareMathOperator{\adj}{adj}
\DeclareMathOperator{\Range}{Im}
\DeclareMathOperator{\mes}{mes}

\tikzstyle{thinarrow}=[densely dashed,very thin]

\title[CLASE-2]
{\mbox{Introducci\'on al Lenguaje de Programaci\'on}\\\mbox{PYTHON}}
\author[juanito55@gmail.com]{\nombre \\
{\small \correo}}
\institute[\iniciales]{\instituto}
\date{\today}

\hypersetup{pdfkeywords={Hermitian Toeplitz matrices,eigenvectors,asymptotics}}

\AtBeginSection[] {
\begin{frame}{Contenido}
\tableofcontents[currentsection]
\end{frame}
}

%===============================================

\begin{document}

\begin{frame}[label=titlepage]
\titlepage

\end{frame}
%===============================================

%\begin{frame}{Contents}
%\tableofcontents

%\end{frame}

%===============================================
% http://docs.python.org.ar/tutorial/2/introduction.html

\section{N\'umeros}

\subsection{N\'umeros}

\begin{frame}{N\'umeros}

$>>>$ ancho = 20

$>>>$ largo = 5*9

$>>>$ ancho * largo

900

%\rule{\textwidth}{1pt}

$>>>$ 3 * 3.75 / 1.5

7.5

$>>>$ 7.0 / 2

3.5

$>>>$ 1j * complex(0,1)

(-1+0j)

$>>>$ 3+1j*3

(3+3j)

$>>>$ (3+1j)*3

(9+3j)

$>>>$ (1+2j)/(1+1j)

(1.5+0.5j)

\end{frame}

\begin{frame}{N\'umeros}

$>>>$ a=1.5+0.5j

$>>>$ a.real

1.5

$>>>$ a.imag

0.5

$>>>$ a=3.0+4.0j

$>>>$ abs(a)  %# sqrt(a.real**2 + a.imag**2)

5.0






\end{frame}

\begin{frame}{N\'umeros Enteros}

$>>>$ impuesto = 12.5 / 100

$>>>$ precio = 100.50

$>>>$ precio * impuesto

12.5625

$>>>$ precio + \_

113.0625

$>>>$ round(\_, 2)

113.06

\end{frame}

\section{Cadenas de caracteres}

\subsection{Cadenas de caracteres}

\begin{frame}{Cadenas de caracteres}

$>>>$ 'huevos y pan'

'huevos y pan'

$>>>$ "doesn't"

"doesn't"

$>>>$ '"Si," le dijo.'

'"Si," le dijo.'

\end{frame}

\begin{frame}{Cadenas de caracteres}

$>>>$ hola = "Esta es una larga cadena que contiene \textbackslash n \textbackslash

varias l\'ineas de texto, tal y como se hace en C. \textbackslash n \textbackslash

    Notar que los espacios en blanco al principio de la linea \textbackslash

 son significantes."

$>>>$ print hola

$>>>$ palabra = 'Ayuda' + 'A'

$>>>$ palabra

'AyudaA'

$>>>$ '$<$' + palabra*5 + '$>$'

'$<$AyudaAAyudaAAyudaAAyudaAAyudaA$>$'

\end{frame}

\begin{frame}{Cadenas de caracteres}

$>>>$ palabra[4]

'a'

$>>>$ palabra[0:2]

'Ay'

$>>>$ palabra[2:4]

'ud'

$>>>$ palabra[:2]    

'Ay'

$>>>$ palabra[2:]    

'udaA'

\end{frame}


\begin{frame}{Cadenas de caracteres}

$>>>$ 'x' + palabra[1:]

'xyudaA'

$>>>$ 'Mas' + palabra[5]

'MasA'

$>>>$ palabra[-1]  
   
'A'

$>>>$ palabra[-2]     

'a'

$>>>$ palabra[-2:]    

'aA'

$>>>$ palabra[:-2]    

'Ayud'


\end{frame}

\section{Listas}

\subsection{Listas}

\begin{frame}{Listas}

$>>>$ a = ['pan', 'huevos', 100, 1234]

$>>>$ a

['pan', 'huevos', 100, 1234]

$>>>$ a[0]

'pan'

$>>>$ a[3]

1234

$>>>$ a[-2]

100

$>>>$ a[1:-1]

['huevos', 100]

$>>>$ a[:2] + ['carne', 2*2]

['pan', 'huevos', 'carne', 4]

$>>>$ 3*a[:3] + ['Boo!']

['pan', 'huevos', 100, 'pan', 'huevos', 100, 'pan', 'huevos', 100, 'Boo!']

\end{frame}

\begin{frame}{Listas}

$>>>$ a

['pan', 'huevos', 100, 1234]

$>>>$ a[2] = a[2] + 23

$>>>$ a

['pan', 'huevos', 123, 1234]

$>>>$ a[0:2] = [1, 12]

$>>>$ a

[1, 12, 123, 1234]

$>>>$ a[0:2] = []

$>>>$ a

[123, 1234]

$>>>$ a[1:1] = ['bruja', 'xyzzy']

$>>>$ a

[123, 'bruja', 'xyzzy', 1234]

$>>>$ a[:0] = a

$>>>$ a

[123, 'bruja', 'xyzzy', 1234, 123, 'bruja', 'xyzzy', 1234]

$>>>$  a[:] = []

$>>>$ a

[]

\end{frame}


\section{Operadores en Python}

\subsection{Operadores l�gicos}

\begin{frame}{Operadores l\'ogicos}

\begin{center}
\[
\begin{tabular}{|c|c|l| }
\hline
 \textbf{Operador} & \textbf{Descripci\'on} &	\textbf{Ejemplo} \\
 \hline
and	& se cumple a y b? & 	r=True and False = r es False \\
\hline
or	& 	se cumple a o b?		&  r=True or False = r es True \\
\hline
not	& 	No a 	& 	r=not True = r es False \\
\hline
 \end{tabular}
\]
\end{center}

\end{frame}

\subsection{Operadores relacionales}

\begin{frame}{Operadores relacionales}

\begin{center}
\[
\begin{tabular}{|c|c|l| }
\hline
 \textbf{Operador} & \textbf{Descripci\'on} &	\textbf{Ejemplo} \\
 \hline
$==$ & Son iguales a y b?  & 	r=5 $== $ 3 = r es False  \\
$!=$  & Son distintos a y b?	  & r=5 $!=$ 3 =  r es True  \\
$<$  & 	Es a menor que b?  & 	r=5 $<$ 3 = r es False \\
$>$  & 	Es a mayor que b?  & 	r=5 $>$ 3 = r es True \\
$<=$  & Es a menor o igual que b?  & 	r=5 $<=$ 5 = r es True \\
$>=$  & Es a mayor o igual que b?  & 	r=5 $>=$ 3 = r es True \\
\hline
 \end{tabular}
\]
\end{center}

\end{frame}

\subsection{Operadores aritm\'eticos}

\begin{frame}{Operadores aritm\'eticos}

\begin{center}
\[
\begin{tabular}{|c|c|l| }
\hline
 \textbf{Operador} & \textbf{Descripci\'on} &	\textbf{Ejemplo} \\
 \hline
$+$ &	Suma	 &	r=3+2 = r es 5  \\
$-$ &		Resta	 &	r=4-7  =  r es -3 \\
$-$ &		Negaci�n	 &	r=-7  =  r es -7 \\
$*$ &		Multiplicaci�n	 &	r=2*6  =  r es 12 \\
$**$	 &	Exponente	 &	r=2**6  = r es 64 \\
$/$ &		Divisi�n	  &	r=3.5/2  =  r es 1.75 \\
$//$	 &	Divisi�n Entera	 &	r=3.5 $//$ 2  = r es 1.0 \\
$\%$	 &	M�dulo	 &	r=7 \% 2 = r es 1 \\
\hline
 \end{tabular}
\]
\end{center}

\end{frame}


\subsection{Operadores a nivel de Bit}

\begin{frame}{Operadores a nivel de Bit}

\begin{center}
\[
\begin{tabular}{|c|c|l| }
\hline
 \textbf{Operador} & \textbf{Descripci\'on} &	\textbf{Ejemplo} \\
 \hline
$\&$ & 	and	 & r=3 \& 2 = r es 2  \\
$|$  & 	or	 & r=3 $|$ 2 = r es 3 \\
.	 & xor	 & r=3 . 2 = r es 1 \\
.	 & not & 	r=.3 = r es -4 \\
$<<$	 & Desplazamiento a la izquierda & 	r=3 $<<$ 1 = r es 6 \\
$>>$	 & Desplazamiento a la derecha	 & r=3 $>>$1 = r es 1 \\
\hline
 \end{tabular}
\]
\end{center}

\end{frame}

\section{Estructuras de Control}

% http://docs.python.org.ar/tutorial/2/controlflow.html

\subsection{La sentencia if}

\begin{frame}{La sentencia \textbf{if}}

$>>>$ x = int(raw\_input("Ingresa un entero, por favor: "))

Ingresa un entero, por favor: 42

$>>>$ if x $<$ 0:

...      x = 0

...      print 'Negativo cambiado a cero'

... elif x $==$ 0:

...      print 'Cero'

... elif x $==$ 1:

...      print 'Simple'

... else:

...      print 'Mas'

...

'Mas'

\end{frame}

\subsection{La sentencia for}

\begin{frame}{La sentencia \textbf{for}}

$>>>$ a = ['gato', 'ventana', 'defenestrado']

$>>>$ for x in a:

...     print x, len(x)

...

gato 4

ventana 7

defenestrado 12

\end{frame}



\end{document}
