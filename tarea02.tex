
\documentclass{article}

% Esto es para poder escribir acentos directamente:
\usepackage[latin1]{inputenc}
% Esto es para que el LaTeX sepa que el texto est� en espa�ol:
\usepackage[spanish]{babel}
\spanishdecimal{.}

% Paquetes de la AMS:
\usepackage{amsmath, amsthm, amsfonts}

% Teoremas
%--------------------------------------------------------------------------
\newtheorem{thm}{Teorema}[section]
\newtheorem{cor}[thm]{Corolario}
\newtheorem{lem}[thm]{Lema}
\newtheorem{prop}[thm]{Proposici�n}
\theoremstyle{definition}
\newtheorem{defn}[thm]{Definici�n}
\theoremstyle{remark}
\newtheorem{rem}[thm]{Observaci�n}

% Atajos.
% Se pueden definir comandos nuevos para acortar cosas que se usan
% frecuentemente. Como ejemplo, aqu� se definen la R y la Z dobles que
% suelen representar a los conjuntos de n�meros reales y enteros.
%--------------------------------------------------------------------------

\def\RR{\mathbb{R}}
\def\ZZ{\mathbb{Z}}

% De la misma forma se pueden definir comandos con argumentos. Por
% ejemplo, aqu� definimos un comando para escribir el valor absoluto
% de algo m�s f�cilmente.
%--------------------------------------------------------------------------
\newcommand{\abs}[1]{\left\vert#1\right\vert}

% Operadores.
% Los operadores nuevos deben definirse como tales para que aparezcan
% correctamente. Como ejemplo definimos en jacobiano:
%--------------------------------------------------------------------------
\DeclareMathOperator{\Jac}{Jac}

%--------------------------------------------------------------------------
\title{Introducci\'on a Lenguaje de Programaci\'on \\
PYTHON}
\author{Juan Per\'ez\\
  \small Dept. de Computo\\
  \small Instituto de Computo\\
  \small juanito@gmail.com
}

\begin{document}
\maketitle

%\abstract{Esto es una plantilla simple para un art�culo en \LaTeX.}

\textbf{TAREA02} \cite{Cd94}.

\hrulefill

\begin{enumerate}
\item Evalue las siguientes expresiones con los siguientes valores: $a = 2.0, b =-3.0, c=5.0$

\begin{enumerate}
\item $a^4 + \frac{1}{b^4} + c =$
\item $a^4 + a^3 + a^2 + b = $
\item $\frac{a-b}{a+c}$
\item $\frac{a-b}{(a+c)^2(a+c)}$
\item $a(b+c)(b-c)$
\item $a^b a^c$
\end{enumerate}

\item hacer un Script para resolver los problemas siguientes.


\begin{enumerate}
\item Calcular el factorial de 10!.
\item Calcular la sumatoria:

\[
1 + 2 + 3 + ... + 100 =
\]

\item Calcular la sumatoria para q=2 y n=100

\[
1 + q + q^2 + q^3 + ... + q^n = 
\]
\end{enumerate}


\end{enumerate}


% Bibliograf�a.
%-----------------------------------------------------------------
\begin{thebibliography}{99}

\bibitem{Cd94} Autor, \emph{T�tulo}, Revista/Editor, (a�o)

\end{thebibliography}

\end{document}